\documentclass{article}

\usepackage{a4wide}
\usepackage{fancyhdr}
\usepackage{graphicx}
\usepackage[ansinew]{inputenc}
\usepackage{amsmath}
\usepackage{amssymb}
\usepackage{url}
\usepackage[british]{babel}


\title{Time-Dependent Schr{\"o}dinger Equation with Magnetic Field}
\author{Etienne Corminboeuf}
\date{\today}

\usepackage[left = 3cm, right = 3cm]{geometry}

\begin{document}

\maketitle

\tableofcontents

\section{Introduction}

\subsection{Mathematic model}

In this report we consider a spinless particle in $\mathbb{R}^d$ with mass $m \in \mathbb{R}_{\geq 0}$ and charge $e\in \mathbb{R}$ in a homogeneous magnetic field $B(t)$. We follow the notation introduced in \cite{paper_orvg} and quickly recapulate that derivation.
In quantum mechanics, the time evolution of a particle subject to a magnetic field is governed by the Pauli equation
\begin{align} \label{eq_pauli}
  i \hbar \partial_t \Psi(x,t) &= H_P(t)\Psi(x,t) \\
  H_P(t) :&= \frac{1}{2m} \sum_{k=1}^d (p_k - e A_k(x,t))^2 + e\phi(x,t) + \tilde{V}(x,t)
\end{align}
where $\tilde{V}(x,t)$ is some external potential.
The magnetic field 2-form $dA$ associated with $B(t)$ is independent of x because of the homogeneity of $B(t)$ and we can thus rewrite the magnetic vector potential to
\begin{equation}
  A(x,t) := \frac{1}{2}B_{jk}(t)x^j \textrm{d}x^k,
\end{equation}
where $B(t)$ = $(B_{jk}(t))_{j,k = 1}^d$ is a real, skew-symmetric matrix. Using the operators
\begin{align}
  L_{jk} & := x_j p_k - x_k p_j \\
  H_B(t) & := - \sum_{j,k = 1}^d B_{jk}(t) L_{jk}
\end{align}
the Pauli-Hamiltonian takes the form
\begin{equation}
  H_P(t) = \frac{1}{2 m} \left(\hbar^{2}(-\Delta)-e \sum_{1 \leqslant j<k \leqslant d} B_{j k}(t) L_{j k} +\frac{e^{2}}{4}\|B(t) x\|_{\mathbb{R}^{d}}^{2}  \right) + e \phi(x,t) + \tilde{V}(x,t).
\end{equation}

\subsection{Numerical model}
We introduce $\epsilon ^2 := \hbar$ and redefine $t$, $x$ and $B$ to find the simplified form
\begin{equation}
  H_P(t) = -\Delta + H_B(t) + V(x,t)
\end{equation}
where $V(x,t) := \frac{1}{2m}\frac{e^{2}}{4}\|B(t) x\|_{\mathbb{R}^{d}}^{2} + e \phi(x,t) + \tilde{V}(x,t)$ is the total potential.
The Schr{\"o}dinger equation is then split up into three separate parts that can be solved numerically.
\begin{align}
  i \epsilon^2 \partial_t \Psi &= -\Delta \Psi \label{num_kin} \tag{K}\\
  i \epsilon^2 \partial_t \Psi &= H_B(t) \Psi \label{num_M} \tag{M}\\
  i \epsilon^2 \partial_t \Psi &= V(x,t) \Psi \label{num_pot} \tag{P}
\end{align}
(\ref{num_kin}) can be solved discretely in Fourier-space and (\ref{num_pot}) by pointwise multiplication with $e^{-i/\epsilon ^2 \int_{t_0}^t dt V(x,t)}$. (\ref{num_M}) is reduced to the linear differential equation
\begin{equation} \label{num_B} \tag{B}
  \frac{d}{dt}y(t) = B(t)y(t)
\end{equation}
\cite{simon_reed} proves the existence of a solution $U(t,t_0)$ to (\ref{num_B}). The unitary representation
\begin{align}
  \rho : SO(d) &\longrightarrow U(L^2(\mathbb{R}^d)) \\
  R &\longmapsto (\rho (R)\Psi)(x) = \Psi(R^{-1} x)
\end{align}
maps the solution $U(t, t_0)$ of (\ref{num_B}) to a solution of (\ref{num_M}). The proof of this statement can be found in \cite{paper_orvg}. The exact flow map $U(t, t_0)$ is approximated through a Magnus expansion proposed by \cite{magnus_integrators}.


\section{Setup and Method}

\section{Results}

\section{Summary and Conclusion}

\begin{thebibliography}{9}
\bibitem{paper_orvg}
  V. Gradinaru and O. Rietmann.
  \textit{A High-Order Integrator for the
          Schr{\"o}odinger Equation with Time-Dependent,
          Homogeneous Magnetic Field}.
  Unpublished article, ETH Z{\"u}rich, submitted for publication, 2020.
\bibitem{simon_reed}
  B. Simon and M. Reed.
  \textit{Fourier Analysis, Self-Adjointness, Volume 2} of \textit{Methods of Modern Mathematical Physics}.
  Academic Press, Boston, 1975.
\bibitem{magnus_integrators}
  S. Blanes and P.C. Moan.
  \textit{Fourth- and sixth-order commutator-free Magnus integrators for linear and non-linear dynamical systems}.
  Applied Numerical Mathematics, 56(12):1519 - 1537, 2006.
\bibitem{constants}
  \textit{Material constants of copper, teflon and beryllium},
  Engineering Toolbox, URL: \url{https://www.engineeringtoolbox.com/}, visited 09.12.2019.
\bibitem{c_te}
  G. T. Furukawa et al.
  \textit{Specific heat capacity of teflon},
  Journal of Research of the National Bureau of Standards. Vol. 49, No. 4 (1952). Page 273 - 279.


\end{thebibliography}

\end{document}
