% Originally developed by Andreas Wieser and Prof. Manfred Einsiedler.

% \documentclass[11pt,a4paper,oneside]{book}

% Default packages and configurations {{{
%%%%%%%%%%%%%%%%%%%%%%%%%%%%%%%%%%%%%%%%%%%%%%%%%%%%%%%%%%%%%%%%%%%%%%%%%%%%%%%%

\usepackage[T1]{fontenc}
\usepackage{amscd}
\usepackage{amsfonts}
\usepackage{amsmath}
\usepackage{amssymb}
\usepackage{amsthm}
% \usepackage{comment}
\usepackage{dsfont}
\usepackage{enumerate}
% \usepackage{faktor}
\usepackage{fancyhdr}


%% STUART BIB URL BREAK
\usepackage{url}
\def\UrlBreaks{\do\/\do-}


\usepackage[hidelinks]{hyperref}
% Colour settings for hyperlink references
% \hypersetup{%
%   colorlinks=false,
%   allcolors=black,
%   pdfborderstyle={/S/U/W 1} % Underline of width 1pt
% }
\hypersetup{%
  colorlinks,
  allcolors=black
}
\usepackage[nameinlink]{cleveref}

\usepackage{indentfirst}
% \usepackage{mathtools}
% \usepackage{url}
\usepackage{verbatim}
% \usepackage{xspace}

% Figures
\usepackage[final]{graphicx}
% \usepackage{caption}
\usepackage{float}
% \usepackage{import}
% \usepackage{pdfpages}
% \usepackage{subcaption}
% Note, graphicx must be imported before tikz
%\usepackage{tikz-cd}
%\usepackage{tikz}
% @legacy, used to have 'quotes' tikzlibrary as well
%\usetikzlibrary{angles,arrows,calc,decorations.pathmorphing,matrix,shapes}
% \usepackage{transparent}
% \usepackage{xifthen}

% }}} %%%%%%%%%%%%%%%%%%%%%%%%%%%%%%%%%%%%%%%%%%%%%%%%%%%%%%%%%%%%%%%%%%%%%%%%%%
% Page Layout {{{
%%%%%%%%%%%%%%%%%%%%%%%%%%%%%%%%%%%%%%%%%%%%%%%%%%%%%%%%%%%%%%%%%%%%%%%%%%%%%%%%
% Setting white space {{{
%%%%%%%%%%%%%%%%%%%%%%%%%%%%%%%%%%%%%%%%%%%%%%%%%%%%%%%%%%%%%%%%%%%%%%%%%%%%%%%%

\usepackage{vmargin}
\RequirePackage[ansinew]{inputenc}
% {L}{T}{R}{B}{Head height}{Head separation}{Foot height}{Foot separation}
\setmarginsrb{1.25in}{0.6in}{1.25in}{0.8in}{20pt}{0.25in}{9pt}{0.3in}

% Set line spacing to 1.25
\usepackage{setspace}
\setstretch{1.2}
\setlength{\parskip}{0.25em}

% Kill indent at first line of new paragraphs
\setlength{\parindent}{15pt}

% Let height of page content vary from page to page (instead of stretching text
% body)
\raggedbottom

% Manage justification rules
\doublehyphendemerits=10000    % No consecutive line hyphens.
\brokenpenalty=10000           % No broken words across columns/pages.
\widowpenalty=9999             % Almost no widows at bottom of page.
\clubpenalty=9999              % Almost no orphans at top of page.
\interfootnotelinepenalty=9999 % Almost never break footnotes.
\binoppenalty=9999             % Almost never break binary operators.
\relpenalty=9999               % Almost never break binary relations.

% }}} %%%%%%%%%%%%%%%%%%%%%%%%%%%%%%%%%%%%%%%%%%%%%%%%%%%%%%%%%%%%%%%%%%%%%%%%%%
% Headers and footers {{{
%%%%%%%%%%%%%%%%%%%%%%%%%%%%%%%%%%%%%%%%%%%%%%%%%%%%%%%%%%%%%%%%%%%%%%%%%%%%%%%%

% Fancy has a header- and footer-rule, see below
\pagestyle{fancy}
% Clear old settings for headers and footers
\fancyhf{}

% Only mark the name of section
% N.b: This will be the last section starting on current page
\renewcommand{\sectionmark}[1]{\markright{#1}}

% Set line width for header and footer
\renewcommand{\headrulewidth}{0.6pt}
\renewcommand{\footrulewidth}{0.6pt}

% Clear old settings
\lhead{}
\rhead{}
\rfoot{\thepage}
\lfoot{}

% Use same settings as prescribed in default pagestyle

\fancypagestyle{plain}{%
  \fancyhf{}                           % Clear old settings
  \fancyfoot[R]{\thepage}              % ...except the right footer
  \renewcommand{\headrulewidth}{0.6pt} % Set line strength for header
  \renewcommand{\footrulewidth}{0.6pt} % Set line strength for footer
}

% Using this for title page
\newcommand{\HRule}{\rule{\linewidth}{0.5mm}}
% }}} %%%%%%%%%%%%%%%%%%%%%%%%%%%%%%%%%%%%%%%%%%%%%%%%%%%%%%%%%%%%%%%%%%%%%%%%%%
% Miscellaneous
%%%%%%%%%%%%%%%%%%%%%%%%%%%%%%%%%%%%%%%%%%%%%%%%%%%%%%%%%%%%%%%%%%%%%%%%%%%%%%%%

% Force footnotes to the bottom of the page
%\usepackage[bottom]{footmisc}
% }}} %%%%%%%%%%%%%%%%%%%%%%%%%%%%%%%%%%%%%%%%%%%%%%%%%%%%%%%%%%%%%%%%%%%%%%%%%%
% Text Body {{{
%%%%%%%%%%%%%%%%%%%%%%%%%%%%%%%%%%%%%%%%%%%%%%%%%%%%%%%%%%%%%%%%%%%%%%%%%%%%%%%%
% Typesetting tools
%%%%%%%%%%%%%%%%%%%%%%%%%%%%%%%%%%%%%%%%%%%%%%%%%%%%%%%%%%%%%%%%%%%%%%%%%%%%%%%%

%\newcounter{lecture}
%\newcommand{\lecture}[1]{%
%  \stepcounter{lecture}
%  \marginpar{\color{orange}\flushleft{\footnotesize Lecture: \thelecture{} #1}}
%}

\usepackage{mdframed}
\newmdenv[
  topline=false,
  bottomline=false,
  rightline=false,
  skipabove=\topsep,
  skipbelow=\topsep,
  innertopmargin=1pt,
  innerbottommargin=1pt,
  skipabove=10pt,
  skipbelow=10pt,
]{siderules}

% }}} %%%%%%%%%%%%%%%%%%%%%%%%%%%%%%%%%%%%%%%%%%%%%%%%%%%%%%%%%%%%%%%%%%%%%%%%%%
% Table of Contents {{{
%%%%%%%%%%%%%%%%%%%%%%%%%%%%%%%%%%%%%%%%%%%%%%%%%%%%%%%%%%%%%%%%%%%%%%%%%%%%%%%%

\usepackage{titlesec}
\usepackage[titles]{tocloft}
% Format section titles in TOC
\renewcommand\cftsecfont{\scshape}
% Format page numbers in TOC
\renewcommand\cftsecpagefont{\normalfont}
% @language Set name of TOC
\AtBeginDocument{\renewcommand\contentsname{Table of Contents}}

% }}} %%%%%%%%%%%%%%%%%%%%%%%%%%%%%%%%%%%%%%%%%%%%%%%%%%%%%%%%%%%%%%%%%%%%%%%%%%
% Bibliography {{{
%%%%%%%%%%%%%%%%%%%%%%%%%%%%%%%%%%%%%%%%%%%%%%%%%%%%%%%%%%%%%%%%%%%%%%%%%%%%%%%%

% Renew \cite command for more control of appearance
\let\defaultcite\cite
\renewcommand{\cite}[2][]{%
  \textnormal{\defaultcite[#1]{#2}}%
}

% }}} %%%%%%%%%%%%%%%%%%%%%%%%%%%%%%%%%%%%%%%%%%%%%%%%%%%%%%%%%%%%%%%%%%%%%%%%%%

% Generic tools {{{
%%%%%%%%%%%%%%%%%%%%%%%%%%%%%%%%%%%%%%%%%%%%%%%%%%%%%%%%%%%%%%%%%%%%%%%%%%%%%%%%

% When referencing lists, often we need two write things such as
% "\textit{(iii)}" which is long. This macro "enumerated reference" makes it
% easier.
% Usage: \eref{n}
\newcommand{\eref}[1]{\textit{(\romannumeral#1)}}

% }}} %%%%%%%%%%%%%%%%%%%%%%%%%%%%%%%%%%%%%%%%%%%%%%%%%%%%%%%%%%%%%%%%%%%%%%%%%%
% Math-Environments {{{
%%%%%%%%%%%%%%%%%%%%%%%%%%%%%%%%%%%%%%%%%%%%%%%%%%%%%%%%%%%%%%%%%%%%%%%%%%%%%%%%

% 1. \newtheorem{<Internal name>}{<External name>}[<env>]
%    Is counted after <env>. e.g.
%    \newtheorem{corollary}{Corollary}[theorem]
%    Theorem 1 is succeeded by Corollary 1.2
% 2. \newtheorem{<Internal name>}[<env>]{<External name>}[<env>]
%    \newtheorem{corollary}[theorem]{Corollary}
%    Is shares counter with <env>. i.e. Theorem 1 is succeeded by Corollary 2

% Declare a new theoremstyle
% \newtheoremstyle{mystyle}%
%    {3pt}                   % Space above
%    {3pt}                   % Space below
%    {\itshape\color{red}}   % Body font
%    {}                      % Indent amount
%    {\bfseries\color{blue}} % Theorem head font
%    {.}                     % Punctuation after theorem head
%    {.5em}                  % Space after theorem head
%    {}                      % Theorem head spec (can be left empty, meaning ‘normal’)

\theoremstyle{plain}

% Base number off chapter if documentclass 'book' is selected, else use
% sections.

\ifundef{\chapter}{%
  \newtheorem{theorem}{Theorem}[section]
  \numberwithin{equation}{section}
  \numberwithin{figure}{section}
}{%
  \newtheorem{theorem}{Theorem}[chapter]
  \numberwithin{equation}{chapter}
  \numberwithin{figure}{chapter}
}

\newtheorem*{theorem*}{Theorem}
\newtheorem{theorem-extra}{Theorem*}

\newtheorem{proposition}[theorem]{Proposition}
\newtheorem*{proposition*}{Proposition}
\newtheorem{proposition-extra}{Proposition*}

\newtheorem{lemma}[theorem]{Lemma}
\newtheorem*{lemma*}{Lemma}
\newtheorem{lemma-extra}{Lemma*}

\newtheorem*{fact}{Fact}

\newtheorem{corollary}[theorem]{Corollary}
\newtheorem*{corollary*}{Corollary}
\newtheorem{generic-corollary}[theorem]{Corollary}

\newtheorem{example}[theorem]{Example}
\newtheorem*{example*}{Example}

\newtheorem{exercise}[theorem]{Exercise}
\newtheorem*{exercise*}{Exercise}
\newtheorem{important-exercise}[theorem]{Important Exercise}

\newtheorem{reminder}[theorem]{Reminder}

\theoremstyle{remark}
  \newtheorem*{remark}{Remark}
\theoremstyle{plain}

\theoremstyle{definition}
  \newtheorem{definition}[theorem]{Definition}
  \newtheorem*{definition*}{Definition}
\theoremstyle{plain}

% }}} %%%%%%%%%%%%%%%%%%%%%%%%%%%%%%%%%%%%%%%%%%%%%%%%%%%%%%%%%%%%%%%%%%%%%%%%%%
% Mathematical notation
%%%%%%%%%%%%%%%%%%%%%%%%%%%%%%%%%%%%%%%%%%%%%%%%%%%%%%%%%%%%%%%%%%%%%%%%%%%%%%%%

% Tool for moving maths expressions
\makeatletter
\newcommand{\raisemath}[1]{\mathpalette{\raisemAth{#1}} }
\newcommand{\raisemAth}[3]{\raisebox{#1}{$#2#3$}}
\makeatother

\newcommand{\define}[1]{\textit{#1}}

% Special Letters {{{
%%%%%%%%%%%%%%%%%%%%%%%%%%%%%%%%%%%%%%%%%%%%%%%%%%%%%%%%%%%%%%%%%%%%%%%%%%%%%%%%

% For faster typesetting of the calligraphic letters
\newcommand{\bA}{\mathbb{A}}
\newcommand{\bB}{\mathbb{B}}
\newcommand{\bC}{\mathbb{C}}
\newcommand{\bD}{\mathbb{D}}
\newcommand{\bE}{\mathbb{E}}
\newcommand{\bF}{\mathbb{F}}
\newcommand{\bG}{\mathbb{G}}
\newcommand{\bH}{\mathbb{H}}
\newcommand{\bI}{\mathbb{I}}
\newcommand{\bJ}{\mathbb{J}}
\newcommand{\bK}{\mathbb{K}}
\newcommand{\bL}{\mathbb{L}}
\newcommand{\bM}{\mathbb{M}}
\newcommand{\bN}{\mathbb{N}}
\newcommand{\bO}{\mathbb{O}}
\newcommand{\bP}{\mathbb{P}}
\newcommand{\bQ}{\mathbb{Q}}
\newcommand{\bR}{\mathbb{R}}
\newcommand{\bS}{\mathbb{S}}
\newcommand{\bT}{\mathbb{T}}
\newcommand{\bU}{\mathbb{U}}
\newcommand{\bV}{\mathbb{V}}
\newcommand{\bW}{\mathbb{W}}
\newcommand{\bX}{\mathbb{X}}
\newcommand{\bY}{\mathbb{Y}}
\newcommand{\bZ}{\mathbb{Z}}

% For faster typesetting of the calligraphic letters
\newcommand{\cA}{\mathcal{A}}
\newcommand{\cB}{\mathcal{B}}
\newcommand{\cC}{\mathcal{C}}
\newcommand{\cD}{\mathcal{D}}
\newcommand{\cE}{\mathcal{E}}
\newcommand{\cF}{\mathcal{F}}
\newcommand{\cG}{\mathcal{G}}
\newcommand{\cH}{\mathcal{H}}
\newcommand{\cI}{\mathcal{I}}
\newcommand{\cJ}{\mathcal{J}}
\newcommand{\cK}{\mathcal{K}}
\newcommand{\cL}{\mathcal{L}}
\newcommand{\cM}{\mathcal{M}}
\newcommand{\cN}{\mathcal{N}}
\newcommand{\cO}{\mathcal{O}}
\newcommand{\cP}{\mathcal{P}}
\newcommand{\cQ}{\mathcal{Q}}
\newcommand{\cR}{\mathcal{R}}
\newcommand{\cS}{\mathcal{S}}
\newcommand{\cT}{\mathcal{T}}
\newcommand{\cU}{\mathcal{U}}
\newcommand{\cV}{\mathcal{V}}
\newcommand{\cW}{\mathcal{W}}
\newcommand{\cX}{\mathcal{X}}
\newcommand{\cY}{\mathcal{Y}}
\newcommand{\cZ}{\mathcal{Z}}

% For faster typesetting of the fraction letters
\newcommand{\fA}{\mathfrak{A}}
\newcommand{\fB}{\mathfrak{B}}
\newcommand{\fC}{\mathfrak{C}}
\newcommand{\fD}{\mathfrak{D}}
\newcommand{\fE}{\mathfrak{E}}
\newcommand{\fF}{\mathfrak{F}}
\newcommand{\fG}{\mathfrak{G}}
\newcommand{\fH}{\mathfrak{H}}
\newcommand{\fI}{\mathfrak{I}}
\newcommand{\fJ}{\mathfrak{J}}
\newcommand{\fK}{\mathfrak{K}}
\newcommand{\fL}{\mathfrak{L}}
\newcommand{\fM}{\mathfrak{M}}
\newcommand{\fN}{\mathfrak{N}}
\newcommand{\fO}{\mathfrak{O}}
\newcommand{\fP}{\mathfrak{P}}
\newcommand{\fQ}{\mathfrak{Q}}
\newcommand{\fR}{\mathfrak{R}}
\newcommand{\fS}{\mathfrak{S}}
\newcommand{\fT}{\mathfrak{T}}
\newcommand{\fU}{\mathfrak{U}}
\newcommand{\fV}{\mathfrak{V}}
\newcommand{\fW}{\mathfrak{W}}
\newcommand{\fX}{\mathfrak{X}}
\newcommand{\fY}{\mathfrak{Y}}
\newcommand{\fZ}{\mathfrak{Z}}

% }}} %%%%%%%%%%%%%%%%%%%%%%%%%%%%%%%%%%%%%%%%%%%%%%%%%%%%%%%%%%%%%%%%%%%%%%%%%%
% Sets {{{
%%%%%%%%%%%%%%%%%%%%%%%%%%%%%%%%%%%%%%%%%%%%%%%%%%%%%%%%%%%%%%%%%%%%%%%%%%%%%%%%

\renewcommand{\subset}{\subseteq}
\renewcommand{\supset}{\supseteq}
\renewcommand{\complement}[1]{{#1}^{c}}
\newcommand{\closure}[1]{\overline{#1}}
\newcommand{\interior}[1]{\mathring{#1}}

% Set
\newcommand{\set}[1]{\left\lbrace#1\right\rbrace}
% Set with conditions (vertical line)
\newcommand{\cset}[2]{\left\lbrace#1\,\mid\,#2\right\rbrace}
% Big Set
\newcommand{\bigset}[1]{\bigl\lbrace#1\bigr\rbrace}
% Big Set with conditions (vertical line)
\newcommand{\bigcset}[2]{\bigl\lbrace#1\,\mid\,#2\bigr\rbrace}
% Tuple
\newcommand{\tup}[1]{\left(#1\right)}

% Characteristic function
\DeclareMathOperator{\one}{\mathds{1}}

% }}} %%%%%%%%%%%%%%%%%%%%%%%%%%%%%%%%%%%%%%%%%%%%%%%%%%%%%%%%%%%%%%%%%%%%%%%%%%
% Trigonometric functions {{{
%%%%%%%%%%%%%%%%%%%%%%%%%%%%%%%%%%%%%%%%%%%%%%%%%%%%%%%%%%%%%%%%%%%%%%%%%%%%%%%%

\DeclareMathOperator{\arsinh}{arsinh}
\DeclareMathOperator{\arcosh}{arcosh}
\DeclareMathOperator{\artanh}{artanh}

% }}} %%%%%%%%%%%%%%%%%%%%%%%%%%%%%%%%%%%%%%%%%%%%%%%%%%%%%%%%%%%%%%%%%%%%%%%%%%
% Analysis {{{
%%%%%%%%%%%%%%%%%%%%%%%%%%%%%%%%%%%%%%%%%%%%%%%%%%%%%%%%%%%%%%%%%%%%%%%%%%%%%%%%

% Derivatives
\DeclareMathOperator{\De}{\thinspace{\rm{D}} }       % Derivative
\DeclareMathOperator{\boldpartial}{\pmb{\partial}}   % (Bold) partial derivative
\DeclareMathOperator{\de}{\thinspace{\rm{d}} }       % Derivative
\DeclareMathOperator{\divergence}{div}               % Divergence
\DeclareMathOperator{\dvol}{\,\operatorname{dvol}}   % Volumetric integration
\DeclareMathOperator{\euler}{\operatorname{e}}       % Euler's constant
\DeclareMathOperator{\graph}{\operatorname{graph}}   % Graph
\DeclareMathOperator{\ii}{\operatorname{i}}          % Imaginary unit
\let\Im\defaultIm{}
\DeclareMathOperator{\Im}{Im}                        % Imaginary part
\DeclareMathOperator{\laplace}{\Delta}               % Laplace operator
\DeclareMathOperator{\metric}{\operatorname{d}}      % Metric
\DeclareMathOperator{\re}{Re}                        % Real part
\DeclareMathOperator{\rot}{rot}                      % Rotation/ curl
\DeclareMathOperator{\tangent}{\operatorname{T}}     % Tangent space
\DeclareMathOperator{\vol}{vol}                      % Volumetric integration
\newcommand{\restr}[2]{\ensuremath{#1\big|_{#2}} }   % Restriction
\newcommand{\abs}[1]{\left\lvert#1\right\rvert}      % Absolute value
\newcommand{\norm}[1]{\left\lVert#1\right\rVert}     % Norm
\newcommand{\support}[1][]{\operatorname{supp}_{#1}} % [Named] Support
\newcommand{\vc}[1]{\mathbf{#1}}                     % Vector
\newcommand{\weak}{\textsuperscript*}                % Weak decoration
\newcommand{\ceiling}[1]{\lceil#1\rceil}             % Ceiling function
\newcommand{\floor}[1]{\lfloor#1\rfloor}             % Floor function

% }}} %%%%%%%%%%%%%%%%%%%%%%%%%%%%%%%%%%%%%%%%%%%%%%%%%%%%%%%%%%%%%%%%%%%%%%%%%%
% Algebra {{{
%%%%%%%%%%%%%%%%%%%%%%%%%%%%%%%%%%%%%%%%%%%%%%%%%%%%%%%%%%%%%%%%%%%%%%%%%%%%%%%%

\DeclareMathOperator{\id}{id}
\DeclareMathOperator{\Tr}{Tr}
\DeclareMathOperator{\Mat}{Mat}
\DeclareMathOperator{\Hom}{Hom}
\DeclareMathOperator{\Aut}{Aut}
\DeclareMathOperator{\ev}{ev}
\let\im\defaultim{}
\DeclareMathOperator{\im}{im}

% Span
\newcommand{\spn}[1]{\ensuremath{\left\langle\,#1\,\right\rangle}}
\DeclareMathOperator{\hull}{Span}
\newcommand{\scalarprod}[2]{\ensuremath{\left\langle#1,#2\right\rangle}}

\DeclareMathOperator{\Char}{char}                    % (Field) Characteristic
\DeclareMathOperator{\Frac}{Frac}                    % Fraction field
\DeclareMathOperator{\Gal}{Gal}                      % Galois group
\DeclareMathOperator{\Obj}{Obj}                      % Object
\DeclareMathOperator{\Orb}{Orb}                      % Orbit
\DeclareMathOperator{\Stab}{Stab}                    % Stabiliser
\DeclareMathOperator{\Sym}{Sym}                      % Symmetric
\DeclareMathOperator{\coker}{coker}                  % Cokernel
\DeclareMathOperator{\irr}{irr}                      % Irreducible polynomial
\DeclareMathOperator{\jac}{Jac}                      % Jacobson Radical
\DeclareMathOperator{\krulldim}{\dim_\mathrm{Krull}} % Krull dimension
\DeclareMathOperator{\lcm}{lcm}                      % Lowest common denominator
\DeclareMathOperator{\nil}{nil}                      % Nilradical
\DeclareMathOperator{\rad}{rad}                      % Jacobson Radical
\DeclareMathOperator{\sgn}{sgn}                      % Sign
\DeclareMathOperator{\tor}{to}                       % Torsion

\newcommand{\Bil}[1][]{\operatorname{Bil}_{#1}}                      % Bilinear maps
\newcommand{\ann}[1][]{\operatorname{ann}_{#1}}                      % Annihilator
\newcommand{\Ass}[1][]{\operatorname{Ass}_{#1}}                      % Associated prime
\newcommand{\height}[1][]{\operatorname{ht}_{#1}}                    % Prime ideal height
\newcommand{\trdeg}[1][]{\operatorname{trdeg}_{#1}}                  % Transcendence degree
\newcommand{\tens}[1][]{\otimes_{\scriptscriptstyle{#1}} }           % Tensor product
\newcommand{\Tens}[1][]{\mathbin{\otimes_{\scriptscriptstyle{#1}} }} % Tensor Product

% Fractions
% Dynamically decide whether to place elements in fraction above each other or
% next to each other based on whether in display mode or not.
% \let\defaultfrac\frac
\newcommand{\newfrac}[2]{\mathchoice{\frac{#1}{#2}}{#1/#2}{#1/#2}{#1/#2}}
\newcommand{\defaultfrac}[2]{\frac{#1}{#2}}

% Quotients
% Dynamically decide how much space should be left between operands and how much
% they should be raised, based on whether in display mode or not.
\newcommand{\quot}[2]{%
  \mathchoice
    {\raisemath{0.4ex}{#1} \big/ \raisemath{-0.4ex}{#2}}%
    {\scalebox{0.9}{$\raisemath{0.1ex}{#1} / \raisemath{-0.3ex}{#2}$} }%
    {#1/#2}%
    {#1/#2}%
}
\newcommand{\q}[2]{\quot{#1}{#2}}

% Class of representatives
\newcommand{\cl}[1]{\overline{#1}}

% Styled arrows
% Overloaded \to for named arrow
\renewcommand{\to}[1][]{\xrightarrow{#1}}

% Injection/ into map
% \newcommand{\into}[1][]{\xhookrightarrow{#1}}
\newcommand{\into}[1][]{\xrightarrow{#1}}

% Surjection/ onto map
% \newcommand{\onto}[1][]{\xrightarrow{#1}\mathrel{\mkern-14mu}\rightarrow}
\newcommand{\onto}[1][]{\xrightarrow{#1}}

% Named homomorphisms
\newcommand{\nHom}[4]{\Hom_{(#1-\mathrm{#2})} \left(#3,\, #4 \right)}

% }}} %%%%%%%%%%%%%%%%%%%%%%%%%%%%%%%%%%%%%%%%%%%%%%%%%%%%%%%%%%%%%%%%%%%%%%%%%%
% Rings, Fields and Groups {{{
%%%%%%%%%%%%%%%%%%%%%%%%%%%%%%%%%%%%%%%%%%%%%%%%%%%%%%%%%%%%%%%%%%%%%%%%%%%%%%%%

\DeclareMathOperator{\A}{\mathbb{A}}
\DeclareMathOperator{\C}{\mathbb{C}}
\DeclareMathOperator{\E}{\mathbb{E}}
\DeclareMathOperator{\F}{\mathbb{F}}
\DeclareMathOperator{\GL}{GL}
\DeclareMathOperator{\K}{\mathbb{K}}
\DeclareMathOperator{\N}{\mathbb{N}}
\let\P\defaultP
\DeclareMathOperator{\P}{\mathbb{P}}
\DeclareMathOperator{\Q}{\mathbb{Q}}
\DeclareMathOperator{\Rk}{\mathbb{R}^k}
\DeclareMathOperator{\Rm}{\mathbb{R}^m}
\DeclareMathOperator{\Rn}{\mathbb{R}^n}
\DeclareMathOperator{\R}{\mathbb{R}}
\DeclareMathOperator{\SL}{SL}
\DeclareMathOperator{\SO}{SO}
\DeclareMathOperator{\Z}{\mathbb{Z}}

% }}} %%%%%%%%%%%%%%%%%%%%%%%%%%%%%%%%%%%%%%%%%%%%%%%%%%%%%%%%%%%%%%%%%%%%%%%%%%
% Computing {{{
%%%%%%%%%%%%%%%%%%%%%%%%%%%%%%%%%%%%%%%%%%%%%%%%%%%%%%%%%%%%%%%%%%%%%%%%%%%%%%%%

% Problems
\DeclareMathOperator{\SAT}{SAT}
\DeclareMathOperator{\CLIQUE}{CLIQUE}

% Time-complexity
\DeclareMathOperator{\Time}{Time}
\DeclareMathOperator{\TIME}{TIME}

% Memory-complexity
\DeclareMathOperator{\Space}{Space}
\DeclareMathOperator{\SPACE}{SPACE}

% Complexity classes
\DeclareMathOperator{\PTIME}{P}
\DeclareMathOperator{\NPTIME}{NP}
\DeclareMathOperator{\DLOG}{DLOG}
\DeclareMathOperator{\PSPACE}{PSPACE}
\DeclareMathOperator{\EXPTIME}{EXPTIME}

% }}} %%%%%%%%%%%%%%%%%%%%%%%%%%%%%%%%%%%%%%%%%%%%%%%%%%%%%%%%%%%%%%%%%%%%%%%%%%
% General Typesetting {{{
%%%%%%%%%%%%%%%%%%%%%%%%%%%%%%%%%%%%%%%%%%%%%%%%%%%%%%%%%%%%%%%%%%%%%%%%%%%%%%%%

\newcommand{\til}[1]{\widetilde{#1}}
% }}} %%%%%%%%%%%%%%%%%%%%%%%%%%%%%%%%%%%%%%%%%%%%%%%%%%%%%%%%%%%%%%%%%%%%%%%%%%
